\begin{frame}
\frametitle{What does FP mean?}
\begin{itemize}
  \item<1-> functional programming is a simple and principled thesis.
  \item<1-> from this thesis, many practical advantages follow.
  \item<2-> the practical consequences \textbf{do not define} functional programming.
  \item<3-> \textbf{all} programs achieve the principle of FP to some extent.
\end{itemize}
\end{frame}

\begin{frame}[fragile]
\frametitle{What does FP mean?}
\framesubtitle{Referential Transparency}
\begin{block}{Placing \lstinline{expression} under test for referential transparency}
\scriptsize
\begin{lstlisting}[mathescape]
result = expression(args)
$\ldots$
arbitrary1(result)
$\ldots$
arbitrary2(result)
\end{lstlisting}
\end{block}
\begin{block}{Refactor the program \textemdash has the program changed?}
\scriptsize
\begin{lstlisting}[mathescape]
$\ldots$
arbitrary1(expression(args))
$\ldots$
arbitrary2(expression(args))
\end{lstlisting}
\end{block}
\end{frame}

\begin{frame}[fragile]
\frametitle{What does FP mean?}
\framesubtitle{Am I functional programming?}
\begin{block}{To what extent does my program exhibit referential transparency?}
\begin{itemize}
  \item<1-> to what extent can I replace \emph{expressions} with their \emph{values}?
  \item<2-> to what extent am I functional programming?
\end{itemize}
\end{block}
\end{frame}

\begin{frame}[fragile]
\frametitle{What does FP mean?}
\framesubtitle{Tools}
\begin{block}{FAQ}
\begin{itemize}
  \item<1-> is (or is not) this programming language a \quote{functional programming language?}
\end{itemize}
\end{block}
\end{frame}
