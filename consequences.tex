\begin{frame}
\frametitle{Consequences of functional programming}
\begin{block}{Reasoning}
\begin{itemize}
\item<1-> since our program expressions are \emph{referentially transparent}, we may reason about each program part independently of all the others.
\item<2-> this idea is called \emph{equational reasoning}.
\item<3-> equational reasoning gives to the ability to comprehend our programs; small or large.
\end{itemize}
\end{block}
\end{frame}

\begin{frame}
\frametitle{Consequences of functional programming}
\begin{block}{Requirements change}
\begin{itemize}
\item<1-> our program solution, at any level, is the composition of smaller, discrete programs \ldots
\item<2-> \ldots only if \emph{referential transparency is preserved}.
\item<3-> if a requirement changes, we need only change those \textbf{independent parts} which correlate to that change.
\end{itemize}
\end{block}
\end{frame}

\begin{frame}
\frametitle{Consequences of functional programming}
\begin{block}{Programs are sub-programs and can be reused}
\begin{itemize}
\item<1-> since programs are (sometimes provably) delineated from others, the opportunity to reuse arises.
\item<2-> functional programming gives rise to exploration of \emph{principled abstraction}.
\end{itemize}
\end{block}
\end{frame}

\begin{frame}[fragile]
\frametitle{Consequences of functional programming}
\begin{block}{Testing in isolation}
\begin{itemize}
\item<1-> since functions do not perform \emph{side-effects}, they can be tested in isolation.
\item<2-> we can perform testing using \emph{universal quantification}
\begin{lstlisting}[style=haskell,mathescape,basicstyle=\scriptsize]
> ((x ++ y) ++ z == x ++ (y ++ z))
OK, passed 100 tests.
\end{lstlisting}
\end{itemize}
\end{block}
\end{frame}

\begin{frame}
\frametitle{Consequences of functional programming}
\begin{block}{Performance}
\begin{itemize}
\item<1-> if programs are made of functions, they may be rearranged arbitrarily without altering the program outcome.
\item<2-> a compiler may rearrange a program structure (but not outcome) to give optimal performance.
\item<3-> existing runtime compilers do this to a small extent e.g. Java VM, .NET CLR
\end{itemize}
\end{block}
\end{frame}

% - code reasoning/equational reasoning
% - composition and moving with changing requirements
% - reuse
% - testing
% performance
% parametricity/proofs
