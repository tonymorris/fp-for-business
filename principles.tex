\begin{frame}
\frametitle{Principles of functional programming}
\begin{block}{Frege's principle of compositionality \cite{janssen2001frege}}
\begin{itemize}
\item<1-> a program is the composition of its constituent programs.
\item<2-> modifying a program is the act of modifying the necessary part.
\item<3-> the concept of a \emph{program part} is \emph{well-formed and measurable} \cite{hughes1989functional}.
\end{itemize}
\end{block}
\end{frame}

\begin{frame}
\frametitle{Principles of functional programming}
\begin{block}{Achieving program composition}
\begin{itemize}
\item<1-> snake-oil sellers will point you at how to achieve program composition.
\item<2-> or more likely, how to manage having failed to achieved it.
          \begin{itemize}
          \item \tiny{object-oriented hoo-haa}
          \item \tiny{agile and lean and ``oh look over there!''}
          \end{itemize}
\item<3-> functional programming is necessary to the goal of composition.
\end{itemize}
\end{block}
\end{frame}

\begin{frame}[fragile]
\frametitle{Principles of functional programming}
\begin{block}{Example}
\begin{lstlisting}[style=java,mathescape,basicstyle=\scriptsize]
if(`player.score` `>` `12`)
  player.setSwizzle(`1000`);
else
  player.setSwizzle(`11`);
\end{lstlisting}
\end{block}
\begin{block}{refactor program}
\begin{lstlisting}[style=java,mathescape,basicstyle=\scriptsize]
player.setSwizzle(`player.score` `>` `12` `?` `1000` `:` `11`);
\end{lstlisting}
\end{block}
\end{frame}

\begin{frame}
\frametitle{Principles of functional programming}
\begin{itemize}
\item<1-> functional programming is the extent to which this program property holds.
\item<2-> \emph{pure} functional programming is when this program property always holds.
          \begin{itemize}
          \item \tiny{including I/O programs}
          \item \tiny{database programs}
          \item \tiny{multi-threaded programs}
          \item \tiny{web applications}
          \end{itemize}  
\end{itemize}
\end{frame}
